%%%%%%%%%%%%%%%%%%%%%%%%%%%%%%%%%%%%%%%%%%%%%%%%%%%%%%%%%%%%%%%%%%%%%%%%%%% 
% 
% Generic template for the anteproyectos of TFC/TFM/TFGs
% 
% $Id: anteproyecto.tex,v 1.6 2018/09/11 12:23:48 macias Exp $
% 
% By:
%  + Javier Macías-Guarasa. 
%    Departamento de Electrónica
%    Universidad de Alcalá
%  + Roberto Barra-Chicote. 
%    Departamento de Ingeniería Electrónica
%    Universidad Politécnica de Madrid   
% 
% Based on original sources by Roberto Barra, Manuel Ocaña, Jesús Nuevo,
% Pedro Revenga, Fernando Herránz and Noelia Hernández. Thanks a lot to
% all of them, and to the many anonymous contributors found (thanks to
% google) that provided help in setting all this up.
% 
% See also the additionalContributors.txt file to check the name of
% additional contributors to this work.
% 
% If you think you can add pieces of relevant/useful examples,
% improvements, please contact us at (macias@depeca.uah.es)
% 
% You can freely use this template and please contribute with
% comments or suggestions!!!
% 
%%%%%%%%%%%%%%%%%%%%%%%%%%%%%%%%%%%%%%%%%%%%%%%%%%%%%%%%%%%%%%%%%%%%%%%%%%% 

% This is for rubber to clean additional files
% rubber: clean anteproyecto.acn anteproyecto.acr anteproyecto.alg anteproyecto.cod anteproyecto.ist anteproyecto.out anteproyecto.sbl anteproyecto.slg anteproyecto.sym anteproyecto.lor

%%%%%%%%%%%%%%%%%%%%%%%%%%%%%%%%%%%%%%%%%%%%%%%%%%%%%%%%%%%%%%%%%%%%%%%%%%% 
% BEGIN Preamble and configuration section
% 
\input{../Config/preamble-anteproyecto.tex}    % DO NOT TOUCH THIS LINE. You can edit
% the file to modify some default settings

%%%%%%%%%%%%%%%%%%%%%%%%%%%%%%%%%%%%%%%%%%%%%%%%%%%%%%%%%%%%%%%%%%%%%%%%%%%
%
% Generic template for TFC/TFM/TFG/Tesis
%
% $Id: myconfig.tex,v 1.39 2020/03/24 17:33:24 macias Exp $
%
% By:
%  + Javier Macías-Guarasa. 
%    Departamento de Electrónica
%    Universidad de Alcalá
%  + Roberto Barra-Chicote. 
%    Departamento de Ingeniería Electrónica
%    Universidad Politécnica de Madrid   
% 
% Based on original sources by Roberto Barra, Manuel Ocaña, Jesús Nuevo,
% Pedro Revenga, Fernando Herránz and Noelia Hernández. Thanks a lot to
% all of them, and to the many anonymous contributors found (thanks to
% google) that provided help in setting all this up.
%
% See also the additionalContributors.txt file to check the name of
% additional contributors to this work.
%
% If you think you can add pieces of relevant/useful examples,
% improvements, please contact us at (macias@depeca.uah.es)
%
% You can freely use this template and please contribute with
% comments or suggestions!!!
%
%%%%%%%%%%%%%%%%%%%%%%%%%%%%%%%%%%%%%%%%%%%%%%%%%%%%%%%%%%%%%%%%%%%%%%%%%%%

%%%%%%%%%%%%%%%%%%%%%%%%%%%%%%%%%%%%%%%%%%%%%%%%%%%%%%%%%%%%%%%%%%%%%%%%%%% 
%
% Contents of this file:
% + Definition of variables controlling compilation flavours
% + Definition of your own commands (samples provided)
%
% You must edit it to suit to your specific case
%
% Specially important are the definition of your variables (title of the
% book, your degree, author name, email, advisors, keywords (in Spanish
% and English), year, ... They will be used in generating the adequate
% front and cover pages, automagically.
%
%%%%%%%%%%%%%%%%%%%%%%%%%%%%%%%%%%%%%%%%%%%%%%%%%%%%%%%%%%%%%%%%%%%%%%%%%%% 

%%%%%%%%%%%%%%%%%%%%%%%%%%%%%%%%%%%%%%%%%%%%%%%%%%%%%%%%%%%%%%%%%%%%%%%%%%% 
% BEGIN Set my own variables (control compilation for different flavours)

% Control language specific modifications
% This can be english or spanish
\newcommand{\mybooklanguage}{spanish}
%\newcommand{\mybooklanguage}{english}

% Control compilation flavour (for PFCs, TFMs, TFGs, Thesis, etc...)
% Degree (titulación), can be:
% IT     - Ingeniería de Telecomunicación
% IE     - Ingeniería Electrónica
% ITTSE  - Ingeniería Técnica de Telecomunicación, Sistemas Electrónicos
% ITTST  - Ingeniería Técnica de Telecomunicación, Sistemas de Telecomunicación
% ITI    - Ingeniería Técnica Industrial, Electrónica Industrial 
% GITT   - Grado en Ingeniería en Tecnologías de la Telecomunicación
% GIEC   - Grado en Ingeniería Electrónica de Comunicaciones
% GIT    - Grado en Ingeniería Telemática
% GIST   - Grado en Ingeniería en Sistemas de Telecomunicación
% GIC    - Grado en Ingeniería de Computadores
% GII    - Grado en Ingeniería Informática
% GSI    - Grado en Sistemas de Información
% GISI   - Grado en Ingeniería en Sistemas de Información
% GIEAI  - Grado en Ingeniería en Electrónica y Automática Industrial
% GITI   - Grado en Ingeniería en Tecnologías Industriales
% MUSEA  - Máster Universitario en Sistemas Electrónicos Avanzados. Sistemas Inteligentes
% MUIT   - Máster Universitario en Ingeniería de Telecomunicación
% MUII   - Máster Universitario en Ingeniería Industrial
% PHDUAH - Doctorado UAH
% PHDUPM - Doctorado UPM
% GEINTRARR - Geintra Research Report (alpha support)
% You can include additional degrees and modify config/myconfig.tex
% config/postamble.tex and cover/cover.tex, generating new specific
% cover files if needed
\newcommand{\mydegree}{GIST}
%\newcommand{\mydegree}{PHDUAH}

\newcommand{\mybookSplittedAdvisors}{true} % if false it will set
                                % "Tutores/Advisors" in the cover
                                % pages. Otherwise it will split in
                                % Tutor/Cotutor Advisor/Co-advisor


\newcommand{\myspecialty}{} % New in TFGs from 20151218!

% General document information
\newcommand{\mybooktitlespanish}{Diseño e implementación de una herramienta online y colaborativa para el análisis temporal de la fiabilidad de encuestas, sondeos y estimación de indicadores}
\newcommand{\mybooktitleenglish}{Unified Template for the Generation of PFCs, TFGs, TFMs and PhD Thesis}

\newcommand{\mybookauthorname}{Francisco Javier}
\newcommand{\mybookauthorsurname}{García López}
\newcommand{\mybookauthor}{\mybookauthorname{} \mybookauthorsurname{}}
%\newcommand{\mybookauthorgender}{female}
\newcommand{\mybookauthorgender}{male}

\newcommand{\mybookdepartment}{Departamento de Electrónica}
\newcommand{\mybookdepartmentEnglish}{Departament of Electronics}
\newcommand{\mybookphdprogram}{Programa de Doctorado en Electrónica: Sistemas Electrónicos Avanzados. Sistemas Inteligentes}
\newcommand{\mybookphdprogramEnglish}{PhD. Program in Electronics: Advanced Electronic Systems. Intelligent Systems}
\newcommand{\mybookresearchgroup}{GEINTRA}
\newcommand{\mybookschool}{Escuela Politécnica Superior}
\newcommand{\mybookuniversity}{Universidad de Alcalá}
\newcommand{\mybookuniversityacronym}{UAH}
\newcommand{\mybookauthordegree}{Ingeniero de Telecomunicación} % Used in UPM
\newcommand{\mybookemail}{macias@depeca.uah.es}

\newcommand{\mybookNameAcademicTutor}{Javier Macías Guarasa} % This is the default in  TFGs from 20151218
\newcommand{\mybookAcademicTutorGender}{male}
\newcommand{\mybookNameCoTutor}{} % In case you need this for yout TF?
\newcommand{\mybookCoTutorGender}{female}

\newcommand{\mybookNameFirstAdvisor}{\mybookNameAcademicTutor} % This is deprecated: set to academic tutor
\newcommand{\mybookNameSecondAdvisor}{\mybookNameCoTutor} % This is deprecated: set to cotutor
\newcommand{\mybookpresident}{Name of the tribunal president}
\newcommand{\mybookfirstvocal}{Name of the first vocal}
\newcommand{\mybooksecondvocal}{Name of the second vocal} % At UAH usually \mybookNameFirstAdvisor
\newcommand{\mybookalternatemember}{Name of the alternate member}
\newcommand{\mybooksecretary}{Name of the secretary (if needed)}

% Calendar dates 
\newcommand{\mybookyear}{2018}

\newcommand{\myanteproyectodate}{6 de enero de 2018}

\newcommand{\mydepositdate}{1 de enero de 2018} % The date you deposit (submit) this document in the Department
\newcommand{\mydepositdateEnglish}{January 1\textsuperscript{st}, 2018} 

% For RR, mydefensedate is date to be shown in the cover
\newcommand{\mydefensedate}{6 de enero de 2018}
\newcommand{\mydefensedateEnglish}{January 6\textsuperscript{th}, 2018}
% If you prefer British English for the date, use this:
% \newcommand{\mydefensedateEnglish}{6\textsuperscript{th} of January, 2018}

\newcommand{\mybookkeywords}{Bsc., Msc. and PhD. Thesis template, \LaTeX, English/Spanish support, automatic generation} % (up to a maximum of five)
\newcommand{\mybookpalabrasclave}{Plantillas de trabajos fin de carrera/máster/grado y tesis doctorales, \LaTeX, soporte de español e inglés, generación automática} % (máximo de cinco)

%\newcommand{\mybookvicerrectorinvestigacion}{Excma. Sra. María Luisa Marina Alegre}
\newcommand{\mybookvicerrectorinvestigacion}{Excmo. Sr. Francisco J. de la Mata de la Mata}
% Por TFGs & TFMs & MUSEA-TFMs paperwork
\newcommand{\mybookdepartmentsecretary}{José Luis Martín Sánchez}
\newcommand{\mybookdateforpaperwork}{22 de mayo de 2019}
\newcommand{\mybookDNIOpenPublishing}{12345678-L} % Required for TFG's & MUSEA-TFMs
                                % paperwork, must be the DNI of the student
\newcommand{\mybookDNIAcademicTutor}{11111111-A}
\newcommand{\mybookDNICotutor}{}
\newcommand{\mybookDNIFirstAdvisor}{\mybookDNIAcademicTutor} % Deprecated: set to that of academic tutor
\newcommand{\mybookDNISecondAdvisor}{\mybookDNICotutor} % Deprecated set to that of cotutor
\newcommand{\mybookFigure}{alumno} % Required
                                % for TFG's: the type of adscription of
                                % the author signing the agreement
                                % (should be "alumno" in most cases)

\newcommand{\mybookresearchreportID}{RR-2018-01}

% Personal details for the anteproyecto request
% Not required in some cases
\newcommand{\mystreet}{C/ Calle de la Calle, 22}
\newcommand{\mycity}{Meco}
\newcommand{\mypostalcode}{28880}
\newcommand{\myprovince}{Madrid}
\newcommand{\mytelephone}{666666666}


% Link color definition
% Color links of the toc/lot/lof entries
%\newcommand{\mytoclinkcolor}{blue}
\newcommand{\mytoclinkcolor}{black}
%\newcommand{\myloflinkcolor}{red}
\newcommand{\myloflinkcolor}{black}
%\newcommand{\mylotlinkcolor}{green}
\newcommand{\mylotlinkcolor}{black}

% This is used in cover/extralistings.tex
%\newcommand{\myothertoclinkcolor}{magenta}
\newcommand{\myothertoclinkcolor}{black}

% Other color links in the document
\newcommand{\mylinkcolor}{blue}
%\newcommand{\mylinkcolor}{black}

% Color links to urls and cites
\newcommand{\myurlcolor}{blue}
%\newcommand{\myurlcolor}{black}
\newcommand{\mycitecolor}{green}
%\newcommand{\mycitecolor}{black}

% END Set my own variables (control compilation for different flavours)
%%%%%%%%%%%%%%%%%%%%%%%%%%%%%%%%%%%%%%%%%%%%%%%%%%%%%%%%%%%%%%%%%%%%%%%%%%% 

%%%%%%%%%%%%%%%%%%%%%%%%%%%%%%%%%%%%%%%%%%%%%%%%%%%%%%%%%%%%%%%%%%%%%%%%%%% 
% BEGIN My bibliography database files
% Define your own commands here

% This should be relative to the path in which book.tex is located
\newcommand{\myreferences}{biblio/biblio}

% END My bibliography database files
%%%%%%%%%%%%%%%%%%%%%%%%%%%%%%%%%%%%%%%%%%%%%%%%%%%%%%%%%%%%%%%%%%%%%%%%%%% 

%%%%%%%%%%%%%%%%%%%%%%%%%%%%%%%%%%%%%%%%%%%%%%%%%%%%%%%%%%%%%%%%%%%%%%%%%%% 
% BEGIN My own commands section 
% Define your own commands here

% This one is to define a specific format for english text in a Spanish
% document
\DeclareRobustCommand{\texten}[1]{\textit{#1}}

\def\ci{\perp\!\!\!\perp}

% Various examples of commonly used commands
\newcommand{\circulo}{\large $\circ$}
\newcommand{\asterisco}{$\ast$}
\newcommand{\cuadrado}{\tiny $\square$}
\newcommand{\triangulo}{\scriptsize $\vartriangle$}
\newcommand{\triangv}{\scriptsize $\triangledown$}
\newcommand{\diamante}{\large $\diamond$}

\newcommand{\new}[1]{\textcolor{magenta}{#1 }}
\newcommand{\argmax}[1]{\underset{#1}{\operatorname{argmax}}}

% This is an example used in the sample chapters
\newcommand{\verticalSpacingSRPMaps}{-0.3cm}

% END My own commands section 
%%%%%%%%%%%%%%%%%%%%%%%%%%%%%%%%%%%%%%%%%%%%%%%%%%%%%%%%%%%%%%%%%%%%%%%%%%% 

%%% Local Variables:
%%% TeX-master: "../book"
%%% End:


    % DO NOT TOUCH THIS LINE, but EDIT THIS FILE 
                                  % to set your specific settings (related
                                  % to the document language, your degree,
                                  % document details (such as title, author
                                  % (you), your email, name of the tribunal
                                  % members, document year, keyword and
                                  % palabras clave) and link colors), and
                                  % define your commonly used commands
                                  % (some examples are provided).

\input{../Config/postamble.tex}   % DO NOT TOUCH THIS LINE. Yes, I know,
                                  % "postamble" is not a valid word... :-)

% path to directories containing images
\graphicspath{{../Book/logos/}{../Book/figures/}{../Book/diagrams/}} % Edit this to your
                                  % needs. Only logos is really required
                                  % when you generate your own content.
% 
% END Preamble and configuration section
%%%%%%%%%%%%%%%%%%%%%%%%%%%%%%%%%%%%%%%%%%%%%%%%%%%%%%%%%%%%%%%%%%%%%%%%%%% 

\title{Anteproyecto de \mybookworktypefull}        % DO NOT TOUCH THIS LINE
\date{\myanteproyectodate}                         % DO NOT TOUCH THIS LINE
\author{\mybookauthor}

%%%%%%%%%%%%%%%%%%%%%%%%%%%%%%%%%%%%%%%%%%%%%%%%%%%%%%%%%%%%%%%%%%%%%%%%%%% 
% Let's start with the real stuff
%%%%%%%%%%%%%%%%%%%%%%%%%%%%%%%%%%%%%%%%%%%%%%%%%%%%%%%%%%%%%%%%%%%%%%%%%%% 
\begin{document}                                   % DO NOT TOUCH THIS LINE

%%%%%%%%%%%%%%%%%%%%%%%%%%%%%%%%%%%%%%%%%%%%%%%%%%%%%%%%%%%%%%%%%%%%%%%%%%% 
% BEGIN within-document configuration, frontpage and cover pages generation
% 

% Set Language dependent issues that must be set after \begin{document}
\input{../Config/setlanguagedependentissues.tex} % DO NOT TOUCH THIS LINE
                                                 % NOR THE FILE

% 
% END within-document configuration, frontpage and cover pages generation
%%%%%%%%%%%%%%%%%%%%%%%%%%%%%%%%%%%%%%%%%%%%%%%%%%%%%%%%%%%%%%%%%%%%%%%%%%% 

\maketitle

\begin{description}                               % DO NOT TOUCH THIS LINE
\item[Título:] \mybooktitlespanish                % DO NOT TOUCH THIS LINE
  \ifthenelse{\equal{\mybooklanguage}{english}}   % DO NOT TOUCH THIS LINE
  {                                               % DO NOT TOUCH THIS LINE
  \item[Título en inglés:] \mybooktitleenglish    % DO NOT TOUCH THIS LINE
  }                                               % DO NOT TOUCH THIS LINE
  {                                               % DO NOT TOUCH THIS LINE
  }                                               % DO NOT TOUCH THIS LINE
\item[Departamento:] \mybookdepartment            % DO NOT TOUCH THIS LINE
\item[\expandafter\makefirstuc\expandafter{\mybookAutorOrAutora}:] \mybookauthor                       % DO NOT TOUCH THIS LINE
\item[\expandafter\makefirstuc\expandafter{\mybookTutorOrTutores}:] \mybookadvisors                   % DO NOT TOUCH THIS LINE
\end{description}                                 % DO NOT TOUCH THIS LINE

%%%%%%%%%%%%%%%%%%%%%%%%%%%%%%%%%%%%%%%%%%%%%%%%%%%%%%%%%%%%%%%%%%%%%%%%%%% 
%%%%%%%%%%%%%%%%%%%%%%%%%%%%%%%%%%%%%%%%%%%%%%%%%%%%%%%%%%%%%%%%%%%%%%%%%%% 
%%%%%%%%%%%%%%%%%%%%%%%%%%%%%%%%%%%%%%%%%%%%%%%%%%%%%%%%%%%%%%%%%%%%%%%%%%% 
%%%%%%%%%%%%%%%%%%%%%%%%%%%%%%%%%%%%%%%%%%%%%%%%%%%%%%%%%%%%%%%%%%%%%%%%%%% 
%%%%%%%%%%%%%%%%%%%%%%%%%%%%%%%%%%%%%%%%%%%%%%%%%%%%%%%%%%%%%%%%%%%%%%%%%%% 
%%%%%%%%%%%%%%%%%%%%%%%%%%%%%%%%%%%%%%%%%%%%%%%%%%%%%%%%%%%%%%%%%%%%%%%%%%% 
%%%%%%%%%%%%%%%%%%%%%%%%%%%%%%%%%%%%%%%%%%%%%%%%%%%%%%%%%%%%%%%%%%%%%%%%%%% 
% BEGIN Normal sections. Edit/modify all within this section

\section{Introducción}
\label{sec:introduccion}

\textit{En este apartado se describirá el contexto en el que se desenvolverá el
  TFG y sus antecedentes si existen. Recuerda que en el anteproyecto
  también se pueden poner referencias bibliográficas como \cite{moore2002}}.


\section{Objetivos y desarrollo}
\label{sec:objetivos-y-campo}

\textit{En este apartado se delimitan y explican con claridad los objetivos
  generales a conseguir con el TFG y la aplicación del mismo, así como los
  objetivos específicos en su caso. Algo del tipo:}

El objetivo fundamental de este proyecto es el diseño, implementación y
evaluación de \ldots

Los objetivos específicos de este proyecto son los siguientes:

\begin{itemize}
\item Realizar un  \ldots
\item Diseñar, implementar y evaluar  \ldots, siguiendo la arquitectura
  mostrada en la figura \ref{fig_arquitectura}, y que tendrá las
  siguientes características:

  \begin{enumerate}
  \item Característica 1 \ldots
  \item Característica 2 \ldots
  \item  \ldots
  \item Característica n \ldots
  \end{enumerate}

\item Analizar  \ldots

\item Documentar  \ldots

\end{itemize}

\begin{figure}[tphb]
  \centering
  \includegraphics[width=3in]{Diagrama2.pdf}
  \caption{Arquitectura del sistema completo.}
  \label{fig_arquitectura}
\end{figure}


\section{Metodología y plan de trabajo}
\label{sec:metodologia-y-plan}

\textit{Aquí se incluirá una descripción (puede ser incluso una enumeración)
  clara de las etapas que se van a seguir, y si es posible se deberá
  incluir un diagrama de Gantt. Por ejemplo algo del estilo a:}

Estas son las fases de desarrollo que se van a seguir para la
consecución de los objetivos del proyecto descritos en la sección~\ref{sec:objetivos-y-campo}:

\begin{enumerate}
  
\item Planificación y análisis (1 mes)
  
  \begin{itemize}
  \item Formación en \ldots
  \item Consulta de la API \ldots
  \item Consulta bibliográfica \ldots
  \item Profundización en herramientas de soporte \ldots
  \item  \ldots
  \end{itemize}

\item Análisis del sistema (0,5 meses)
  \begin{itemize}
  \item Se analizan los plazos y tiempos respecto a la planificación del análisis.
  \item Si no se ajustan al tiempo disponible se deberá modificar o adaptar el diseño a los tiempos.
  \end{itemize}


\item Diseño, implementación y evaluación del  \ldots (2 meses)
  \begin{itemize}
  \item Definición del  \ldots
  \item Implementación del  \ldots
  \item Evaluación del  \ldots
  \end{itemize}
  
\item Implementación y evaluación de  \ldots (1 mes)

\item Diseño, implementación y evaluación del módulo  \ldots (1 mes):
  \begin{itemize}
  \item Definición de  \ldots
  \item Definición de  \ldots
  \item Implementación y evaluación  \ldots
  \end{itemize}

\item Integración y  \ldots (1 mes)

\item Documentación  \ldots (0,5 meses)

\end{enumerate}


\section{Medios}
\label{sec:medios}

\textit{Aquí se describen de los medios necesarios para realizar el TFG. Por
  ejemplo:}

Las herramientas que van a ser necesarias para desarrollar este proyecto
son las siguientes:

\begin{itemize}
\item PC compatible
\item Sensor  \ldots
\item Sistema operativo GNU/Linux~\cite{gnulinux}
\item Entorno de desarrollo Emacs/Vim~\cite{emacs}
\item Procesador de textos \LaTeX~\cite{lamport94}
\item Control de versiones CVS~\cite{cvs}
\item Compilador C/C++ gcc~\cite{gcc}
\item Gestor de compilaciones make~\cite{make}
\item Robot con movilidad.
\item  \ldots
\end{itemize}



Otros recursos necesarios para la elaboración del proyecto son:

\begin{itemize}
\item Herramientas  \ldots
\item Sistema de desarrollo  \ldots
\item  \ldots
\end{itemize}




% 
% END Normal sections. Edit/modify all within this section
%%%%%%%%%%%%%%%%%%%%%%%%%%%%%%%%%%%%%%%%%%%%%%%%%%%%%%%%%%%%%%%%%%%%%%%%%%% 
%%%%%%%%%%%%%%%%%%%%%%%%%%%%%%%%%%%%%%%%%%%%%%%%%%%%%%%%%%%%%%%%%%%%%%%%%%% 
%%%%%%%%%%%%%%%%%%%%%%%%%%%%%%%%%%%%%%%%%%%%%%%%%%%%%%%%%%%%%%%%%%%%%%%%%%% 
%%%%%%%%%%%%%%%%%%%%%%%%%%%%%%%%%%%%%%%%%%%%%%%%%%%%%%%%%%%%%%%%%%%%%%%%%%% 
%%%%%%%%%%%%%%%%%%%%%%%%%%%%%%%%%%%%%%%%%%%%%%%%%%%%%%%%%%%%%%%%%%%%%%%%%%% 
%%%%%%%%%%%%%%%%%%%%%%%%%%%%%%%%%%%%%%%%%%%%%%%%%%%%%%%%%%%%%%%%%%%%%%%%%%% 
%%%%%%%%%%%%%%%%%%%%%%%%%%%%%%%%%%%%%%%%%%%%%%%%%%%%%%%%%%%%%%%%%%%%%%%%%%% 


%%%%%%%%%%%%%%%%%%%%%%%%%%%%%%%%%%%%%%%%%%%%%%%%%%%%%%%%%%%%%%%%%%%%%%%%%%% 
% Bibliography
%%%%%%%%%%%%%%%%%%%%%%%%%%%%%%%%%%%%%%%%%%%%%%%%%%%%%%%%%%%%%%%%%%%%%%%%%%% 
\input{../Book/biblio/bibliography.tex}               % EDIT this file if required



\end{document}

