%%%%%%%%%%%%%%%%%%%%%%%%%%%%%%%%%%%%%%%%%%%%%%%%%%%%%%%%%%%%%%%%%%%%%%%%%%% 
% 
% Generic template for the anteproyectos of TFC/TFM/TFGs
% 
% $Id: anteproyecto.tex,v 1.6 2018/09/11 12:23:48 macias Exp $
% 
% By:
%  + Javier Macías-Guarasa. 
%    Departamento de Electrónica
%    Universidad de Alcalá
%  + Roberto Barra-Chicote. 
%    Departamento de Ingeniería Electrónica
%    Universidad Politécnica de Madrid   
% 
% Based on original sources by Roberto Barra, Manuel Ocaña, Jesús Nuevo,
% Pedro Revenga, Fernando Herránz and Noelia Hernández. Thanks a lot to
% all of them, and to the many anonymous contributors found (thanks to
% google) that provided help in setting all this up.
% 
% See also the additionalContributors.txt file to check the name of
% additional contributors to this work.
% 
% If you think you can add pieces of relevant/useful examples,
% improvements, please contact us at (macias@depeca.uah.es)
% 
% You can freely use this template and please contribute with
% comments or suggestions!!!
% 
%%%%%%%%%%%%%%%%%%%%%%%%%%%%%%%%%%%%%%%%%%%%%%%%%%%%%%%%%%%%%%%%%%%%%%%%%%% 

% This is for rubber to clean additional files
% rubber: clean anteproyecto.acn anteproyecto.acr anteproyecto.alg anteproyecto.cod anteproyecto.ist anteproyecto.out anteproyecto.sbl anteproyecto.slg anteproyecto.sym anteproyecto.lor

%%%%%%%%%%%%%%%%%%%%%%%%%%%%%%%%%%%%%%%%%%%%%%%%%%%%%%%%%%%%%%%%%%%%%%%%%%% 
% BEGIN Preamble and configuration section
% 
\input{../Config/preamble-anteproyecto.tex}    % DO NOT TOUCH THIS LINE. You can edit
% the file to modify some default settings

\input{../Config/myconfig.tex}    % DO NOT TOUCH THIS LINE, but EDIT THIS FILE 
                                  % to set your specific settings (related
                                  % to the document language, your degree,
                                  % document details (such as title, author
                                  % (you), your email, name of the tribunal
                                  % members, document year, keyword and
                                  % palabras clave) and link colors), and
                                  % define your commonly used commands
                                  % (some examples are provided).

\input{../Config/postamble.tex}   % DO NOT TOUCH THIS LINE. Yes, I know,
                                  % "postamble" is not a valid word... :-)

% path to directories containing images
\graphicspath{{../Book/logos/}{../Book/figures/}{../Book/diagrams/}} % Edit this to your
                                  % needs. Only logos is really required
                                  % when you generate your own content.
% 
% END Preamble and configuration section
%%%%%%%%%%%%%%%%%%%%%%%%%%%%%%%%%%%%%%%%%%%%%%%%%%%%%%%%%%%%%%%%%%%%%%%%%%% 

\title{Anteproyecto de \mybookworktypefull}        % DO NOT TOUCH THIS LINE
\date{\myanteproyectodate}                         % DO NOT TOUCH THIS LINE
\author{\mybookauthor}

%%%%%%%%%%%%%%%%%%%%%%%%%%%%%%%%%%%%%%%%%%%%%%%%%%%%%%%%%%%%%%%%%%%%%%%%%%% 
% Let's start with the real stuff
%%%%%%%%%%%%%%%%%%%%%%%%%%%%%%%%%%%%%%%%%%%%%%%%%%%%%%%%%%%%%%%%%%%%%%%%%%% 
\begin{document}                                   % DO NOT TOUCH THIS LINE

%%%%%%%%%%%%%%%%%%%%%%%%%%%%%%%%%%%%%%%%%%%%%%%%%%%%%%%%%%%%%%%%%%%%%%%%%%% 
% BEGIN within-document configuration, frontpage and cover pages generation
% 

% Set Language dependent issues that must be set after \begin{document}
\input{../Config/setlanguagedependentissues.tex} % DO NOT TOUCH THIS LINE
                                                 % NOR THE FILE

% 
% END within-document configuration, frontpage and cover pages generation
%%%%%%%%%%%%%%%%%%%%%%%%%%%%%%%%%%%%%%%%%%%%%%%%%%%%%%%%%%%%%%%%%%%%%%%%%%% 

\maketitle

\begin{description}                               % DO NOT TOUCH THIS LINE
\item[Título:] \mybooktitlespanish                % DO NOT TOUCH THIS LINE
  \ifthenelse{\equal{\mybooklanguage}{english}}   % DO NOT TOUCH THIS LINE
  {                                               % DO NOT TOUCH THIS LINE
  \item[Título en inglés:] \mybooktitleenglish    % DO NOT TOUCH THIS LINE
  }                                               % DO NOT TOUCH THIS LINE
  {                                               % DO NOT TOUCH THIS LINE
  }                                               % DO NOT TOUCH THIS LINE
\item[Departamento:] \mybookdepartment            % DO NOT TOUCH THIS LINE
\item[\expandafter\makefirstuc\expandafter{\mybookAutorOrAutora}:] \mybookauthor                       % DO NOT TOUCH THIS LINE
\item[\expandafter\makefirstuc\expandafter{\mybookTutorOrTutores}:] \mybookadvisors                   % DO NOT TOUCH THIS LINE
\end{description}                                 % DO NOT TOUCH THIS LINE

%%%%%%%%%%%%%%%%%%%%%%%%%%%%%%%%%%%%%%%%%%%%%%%%%%%%%%%%%%%%%%%%%%%%%%%%%%% 
%%%%%%%%%%%%%%%%%%%%%%%%%%%%%%%%%%%%%%%%%%%%%%%%%%%%%%%%%%%%%%%%%%%%%%%%%%% 
%%%%%%%%%%%%%%%%%%%%%%%%%%%%%%%%%%%%%%%%%%%%%%%%%%%%%%%%%%%%%%%%%%%%%%%%%%% 
%%%%%%%%%%%%%%%%%%%%%%%%%%%%%%%%%%%%%%%%%%%%%%%%%%%%%%%%%%%%%%%%%%%%%%%%%%% 
%%%%%%%%%%%%%%%%%%%%%%%%%%%%%%%%%%%%%%%%%%%%%%%%%%%%%%%%%%%%%%%%%%%%%%%%%%% 
%%%%%%%%%%%%%%%%%%%%%%%%%%%%%%%%%%%%%%%%%%%%%%%%%%%%%%%%%%%%%%%%%%%%%%%%%%% 
%%%%%%%%%%%%%%%%%%%%%%%%%%%%%%%%%%%%%%%%%%%%%%%%%%%%%%%%%%%%%%%%%%%%%%%%%%% 
% BEGIN Normal sections. Edit/modify all within this section

\section{Introducción}
\label{sec:introduccion}

Todo sondeo o estimación que se publica en los distintos medios suele tener un impacto y consumo temporal, que suele ir actualizándose con otros sondeos regularmente o tal como se acerca cierto acontecimiento. Por ejemplo, un informe de estimación de crecimiento económico se publica regularmente, mientras que los sondeos electorales terminan cuando se realiza una votación. En ambos casos, son estudios que alimentan a medios de comunicación hasta que tienen nuevos sondeos o ya unos resultados finales.\\
Lo que busca este proyecto es permitir realizar un estudio y análisis temporal de las predicciones de esos estudios con respecto a unos resultados finales, y de esa forma poder estimar qué fiabilidad han tenido.\\
Los usuarios podrán colaborar con la herramienta online personalizando sondeos a seguir o analizar, así como alimentar las bases de datos con estimaciones de nuevas fuentes, y, por supuesto, consultar toda la información ya disponible y analizada en la herramienta.



\section{Objetivos y desarrollo}
\label{sec:objetivos-y-campo}

El objetivo fundamental de este proyecto es el diseño, implementación y evaluación de una herramienta online y colaborativa para el análisis temporal de la fiabilidad de encuestas, sondeos y estimación de indicadores.\\
La herramienta online deberá ser totalmente adaptable y configurable a cualquier tipo análisis, permitiendo tanto introducir la información de forma manual, como ser capaz de extraerla de forma automática de distintas fuentes.\\
Los usuarios podrán cargar sus fuentes de información y todas en conjunto realizar un seguimiento que permita analizar la fiabilidad de las predicciones de los estudios previos considerados.

Los objetivos específicos de este proyecto son los siguientes:

\begin{itemize}
\item Realizar un estudio para identificar la mejor estrategia de diseño de la estructura de datos de soporte y la lógica de la aplicación, de forma que pueda ser totalmente adaptable a cualquier tipo de sondeo, encuesta o estimación de indicadorers.
\item Diseñar, implementar y evaluar una aplicación web, que tendrá las
  siguientes características:

  \begin{enumerate}
  \item Permitir la configuración de la información de forma que sea fácilmente utilizable por la lógica de la aplicación y facilite los procedimientos de interpretación de los datos.
  \item Permitir la definición versátil de cualquier elemento que vaya a ser analizado, fundamentalmente encuestas y sondeos, así como predicción de indicadores.
  \item Permitir configurar distintos tipos de usuarios, cada uno con roles bien definidos.
  \item Permitir la introducir manual de predicciones y resultados  de los elementos considerados.
  \item Permitir la captura automática de predicciones y resultados de fuentes determinadas, a partir de documentos de texto, enlaces web, imágenes, etc.
  \item Proporcionar una interfaz web cómoda y potente que permita la consulta de los datos introducidos y la realización de un seguimiento exhaustivo de cada uno de los elementos considerados.
  \end{enumerate}

\item Analizar la forma más eficiente de desarrollar la herramienta online, teniendo en cuenta la importancia que tendrá la parte visual e interactiva de la misma; así como estudiar de qué forma se ha de guardar y relacionar toda la información que se vaya añadiendo.

\item Documentar las distintas fases del ciclo de vida del software, que se explican en la sección~\ref{sec:metodologia-y-plan}

\end{itemize}


\section{Metodología y plan de trabajo}
\label{sec:metodologia-y-plan}

Estas son las fases de desarrollo que se van a seguir para la
consecución de los objetivos del proyecto descritos en la sección~\ref{sec:objetivos-y-campo}:

\begin{enumerate}
  
\item Requisitos (2 semanas)
  \begin{itemize}
  \item Identificar exactamente qué se desea mostrar y cómo, con respecto al planteamiento inicial de objetivos.
  \item A partir del planteamiento inicial, diseñar la estructura de datos y el diagrama de bloques de la lógica de la aplicación para diferenciar la separación entre las distintas utilidades que ofrecerá la plataforma.
  \item Identificar las funcionalidades a las que podrá acceder cada usuario.
  \item Determinar cuál es la mejor forma para desarrollar la herramienta, si un software de escritorio que envía y recibe información, una aplicación web multiusuario, etc., teniendo en cuenta que se primará el acceso web para todo el proceso de interacción con la plataforma.
  \end{itemize}

\item Análisis (4 semanas)
  \begin{itemize}
  \item Evaluar los plazos y tiempos respecto a la planificación del análisis, y adaptarlos al tiempo disponible, modificando o adaptando el diseño previsto.
  \item Identificar qué patrón de arquitectura se adec\'ua mejor (MVC o MVVM).
  \item Definir qué tipo de base de datos es la que mejor se ajusta a los datos que van a ser tratados.
  \item Definir los roles de usuarios y la funcionalidad disponible para cada uno de ellos.
  \item Definir las funciones de cada módulo de la plataforma y cómo se relacionan entre sí.
  \item Definir cómo se van a identificar los distintos sondeos o encuestas a seguir para que sea configurable por cada usuario y empleado por otros para realizar la introducción de información desde distintas fuentes.
  \item Definir cómo actuarán y se relacionarán los distintos módulos para obtener información, ya sea analizando enlaces web, ficheros de texto, imágenes, etc. Las distintas fuentes han de alimentar la misma base que relacionará los datos.
  \item Definir el framework a utilizar.
  \end{itemize}


\item Diseño (3 semanas)
  \begin{itemize}
  \item Diseño de la arquitectura general de la plataforma
  \item Diseño particular de las distintas utilidades de la web
  \item Diseño de la base de datos
  \end{itemize}
  
\item Codificaci\'on e integraci\'on (5 semanas)
  \begin{itemize}
  \item Crea la estructura de la base de datos diseñada y que se integrará con la aplicación web
  \item Implementar toda la aplicación web en base al análisis previo
  \end{itemize}

\item Pruebas (1 semana)
  \begin{itemize}
  \item Se probará que todos los módulos funcionan correctamente y que todas las funcionalidades estén presentes
  \end{itemize}

\item Despliegue  (1 semana)
  \begin{itemize}
  \item Para permitir pruebas finales en un entorno real.
  \end{itemize}

\end{enumerate}


\section{Medios}
\label{sec:medios}

Las herramientas que van a ser necesarias para desarrollar este proyecto
son las siguientes:

\begin{itemize}
\item PC compatible
\item Sistema operativo windows~\cite{windows}
\item Entorno de desarrollo compatible con .NET Core ~\cite{dotnetwhat} ~\cite{dotnet}
\item Procesador de textos \LaTeX~\cite{lamport94}
\item Software de control de versiones~\cite{git}
\item Base de datos relacional
\item Librerías para OCR, scraping, etc. dependiendo de las necesidades de las fuentes de información.
\end{itemize}




% 
% END Normal sections. Edit/modify all within this section
%%%%%%%%%%%%%%%%%%%%%%%%%%%%%%%%%%%%%%%%%%%%%%%%%%%%%%%%%%%%%%%%%%%%%%%%%%% 
%%%%%%%%%%%%%%%%%%%%%%%%%%%%%%%%%%%%%%%%%%%%%%%%%%%%%%%%%%%%%%%%%%%%%%%%%%% 
%%%%%%%%%%%%%%%%%%%%%%%%%%%%%%%%%%%%%%%%%%%%%%%%%%%%%%%%%%%%%%%%%%%%%%%%%%% 
%%%%%%%%%%%%%%%%%%%%%%%%%%%%%%%%%%%%%%%%%%%%%%%%%%%%%%%%%%%%%%%%%%%%%%%%%%% 
%%%%%%%%%%%%%%%%%%%%%%%%%%%%%%%%%%%%%%%%%%%%%%%%%%%%%%%%%%%%%%%%%%%%%%%%%%% 
%%%%%%%%%%%%%%%%%%%%%%%%%%%%%%%%%%%%%%%%%%%%%%%%%%%%%%%%%%%%%%%%%%%%%%%%%%% 
%%%%%%%%%%%%%%%%%%%%%%%%%%%%%%%%%%%%%%%%%%%%%%%%%%%%%%%%%%%%%%%%%%%%%%%%%%% 


%%%%%%%%%%%%%%%%%%%%%%%%%%%%%%%%%%%%%%%%%%%%%%%%%%%%%%%%%%%%%%%%%%%%%%%%%%% 
% Bibliography
%%%%%%%%%%%%%%%%%%%%%%%%%%%%%%%%%%%%%%%%%%%%%%%%%%%%%%%%%%%%%%%%%%%%%%%%%%% 
\input{../Book/biblio/bibliography.tex}               % EDIT this file if required



\end{document}

