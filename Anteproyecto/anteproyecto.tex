%%%%%%%%%%%%%%%%%%%%%%%%%%%%%%%%%%%%%%%%%%%%%%%%%%%%%%%%%%%%%%%%%%%%%%%%%%% 
% 
% Generic template for the anteproyectos of TFC/TFM/TFGs
% 
% $Id: anteproyecto.tex,v 1.6 2018/09/11 12:23:48 macias Exp $
% 
% By:
%  + Javier Macías-Guarasa. 
%    Departamento de Electrónica
%    Universidad de Alcalá
%  + Roberto Barra-Chicote. 
%    Departamento de Ingeniería Electrónica
%    Universidad Politécnica de Madrid   
% 
% Based on original sources by Roberto Barra, Manuel Ocaña, Jesús Nuevo,
% Pedro Revenga, Fernando Herránz and Noelia Hernández. Thanks a lot to
% all of them, and to the many anonymous contributors found (thanks to
% google) that provided help in setting all this up.
% 
% See also the additionalContributors.txt file to check the name of
% additional contributors to this work.
% 
% If you think you can add pieces of relevant/useful examples,
% improvements, please contact us at (macias@depeca.uah.es)
% 
% You can freely use this template and please contribute with
% comments or suggestions!!!
% 
%%%%%%%%%%%%%%%%%%%%%%%%%%%%%%%%%%%%%%%%%%%%%%%%%%%%%%%%%%%%%%%%%%%%%%%%%%% 

% This is for rubber to clean additional files
% rubber: clean anteproyecto.acn anteproyecto.acr anteproyecto.alg anteproyecto.cod anteproyecto.ist anteproyecto.out anteproyecto.sbl anteproyecto.slg anteproyecto.sym anteproyecto.lor

%%%%%%%%%%%%%%%%%%%%%%%%%%%%%%%%%%%%%%%%%%%%%%%%%%%%%%%%%%%%%%%%%%%%%%%%%%% 
% BEGIN Preamble and configuration section
% 
\input{../Config/preamble-anteproyecto.tex}    % DO NOT TOUCH THIS LINE. You can edit
% the file to modify some default settings

\input{../Config/myconfig.tex}    % DO NOT TOUCH THIS LINE, but EDIT THIS FILE 
                                  % to set your specific settings (related
                                  % to the document language, your degree,
                                  % document details (such as title, author
                                  % (you), your email, name of the tribunal
                                  % members, document year, keyword and
                                  % palabras clave) and link colors), and
                                  % define your commonly used commands
                                  % (some examples are provided).

\input{../Config/postamble.tex}   % DO NOT TOUCH THIS LINE. Yes, I know,
                                  % "postamble" is not a valid word... :-)

% path to directories containing images
\graphicspath{{../Book/logos/}{../Book/figures/}{../Book/diagrams/}} % Edit this to your
                                  % needs. Only logos is really required
                                  % when you generate your own content.
% 
% END Preamble and configuration section
%%%%%%%%%%%%%%%%%%%%%%%%%%%%%%%%%%%%%%%%%%%%%%%%%%%%%%%%%%%%%%%%%%%%%%%%%%% 

\title{Anteproyecto de \mybookworktypefull}        % DO NOT TOUCH THIS LINE
\date{\myanteproyectodate}                         % DO NOT TOUCH THIS LINE
\author{\mybookauthor}

%%%%%%%%%%%%%%%%%%%%%%%%%%%%%%%%%%%%%%%%%%%%%%%%%%%%%%%%%%%%%%%%%%%%%%%%%%% 
% Let's start with the real stuff
%%%%%%%%%%%%%%%%%%%%%%%%%%%%%%%%%%%%%%%%%%%%%%%%%%%%%%%%%%%%%%%%%%%%%%%%%%% 
\begin{document}                                   % DO NOT TOUCH THIS LINE

%%%%%%%%%%%%%%%%%%%%%%%%%%%%%%%%%%%%%%%%%%%%%%%%%%%%%%%%%%%%%%%%%%%%%%%%%%% 
% BEGIN within-document configuration, frontpage and cover pages generation
% 

% Set Language dependent issues that must be set after \begin{document}
\input{../Config/setlanguagedependentissues.tex} % DO NOT TOUCH THIS LINE
                                                 % NOR THE FILE

% 
% END within-document configuration, frontpage and cover pages generation
%%%%%%%%%%%%%%%%%%%%%%%%%%%%%%%%%%%%%%%%%%%%%%%%%%%%%%%%%%%%%%%%%%%%%%%%%%% 

\maketitle

\begin{description}                               % DO NOT TOUCH THIS LINE
\item[Título:] \mybooktitlespanish                % DO NOT TOUCH THIS LINE
  \ifthenelse{\equal{\mybooklanguage}{english}}   % DO NOT TOUCH THIS LINE
  {                                               % DO NOT TOUCH THIS LINE
  \item[Título en inglés:] \mybooktitleenglish    % DO NOT TOUCH THIS LINE
  }                                               % DO NOT TOUCH THIS LINE
  {                                               % DO NOT TOUCH THIS LINE
  }                                               % DO NOT TOUCH THIS LINE
\item[Departamento:] \mybookdepartment            % DO NOT TOUCH THIS LINE
\item[\expandafter\makefirstuc\expandafter{\mybookAutorOrAutora}:] \mybookauthor                       % DO NOT TOUCH THIS LINE
\item[\expandafter\makefirstuc\expandafter{\mybookTutorOrTutores}:] \mybookadvisors                   % DO NOT TOUCH THIS LINE
\end{description}                                 % DO NOT TOUCH THIS LINE

%%%%%%%%%%%%%%%%%%%%%%%%%%%%%%%%%%%%%%%%%%%%%%%%%%%%%%%%%%%%%%%%%%%%%%%%%%% 
%%%%%%%%%%%%%%%%%%%%%%%%%%%%%%%%%%%%%%%%%%%%%%%%%%%%%%%%%%%%%%%%%%%%%%%%%%% 
%%%%%%%%%%%%%%%%%%%%%%%%%%%%%%%%%%%%%%%%%%%%%%%%%%%%%%%%%%%%%%%%%%%%%%%%%%% 
%%%%%%%%%%%%%%%%%%%%%%%%%%%%%%%%%%%%%%%%%%%%%%%%%%%%%%%%%%%%%%%%%%%%%%%%%%% 
%%%%%%%%%%%%%%%%%%%%%%%%%%%%%%%%%%%%%%%%%%%%%%%%%%%%%%%%%%%%%%%%%%%%%%%%%%% 
%%%%%%%%%%%%%%%%%%%%%%%%%%%%%%%%%%%%%%%%%%%%%%%%%%%%%%%%%%%%%%%%%%%%%%%%%%% 
%%%%%%%%%%%%%%%%%%%%%%%%%%%%%%%%%%%%%%%%%%%%%%%%%%%%%%%%%%%%%%%%%%%%%%%%%%% 
% BEGIN Normal sections. Edit/modify all within this section

\section{Introducción}
\label{sec:introduccion}

Todo sondeo que se publica en los distintos medios suele tener un impacto temporal justo en el momento en el que es publicado, y a la vez es olvidado cuando nuevos sondeos son publicados y finalmente desechados una vez ya han aparecido los resultados finales. Lo que se busca con este proyecto es responder a la pregunta de qué cerca o fiables habían sido esas encuestas o sondeos respecto a lo que finalmente ha sido el resultado final de, por ejemplo, unas votaciones electorales.
Con esta herramienta se podrá responder a esa pregunta y hacer un seguimiento de las distintas fuentes, y al ser colaborativa, los propios usuario podrán interaccionar y añadir los propios sondeos que quieran comprobar. 


\section{Objetivos y desarrollo}
\label{sec:objetivos-y-campo}

El objetivo fundamental de este proyecto es el diseño, implementación y evaluación de una herramienta online y colaborativa para el análisis temporal de la fiabilidad de encuestas, sondeos y estimación de indicadores.\\
La aplicación web deberá ser totalmente adaptable y configurable a cualquier tipo análisis, permitiendo tanto introducir la información de forma manual, así como ser capaz de extraerla de forma automática de distintas fuentes.\\
Los usuarios podrán cargar sus fuentes de información y todas en conjunto realizar un seguimiento que permita observar la fiabilidad.

Los objetivos específicos de este proyecto son los siguientes:

\begin{itemize}
\item Realizar un estudio para identificar la forma más optima de ordenar toda la información, de forma que pueda ser totalmente adaptable a todo tipo de sondeo o encuesta.
\item Diseñar, implementar y evaluar una aplicación web, y que tendrá las
  siguientes características:

  \begin{enumerate}
  \item Permitir consultar los datos introducidos y realizar un seguimiento.
  \item Configurar la información de forma que sea interpretable.
  \item Introducir datos manualmente.
  \item Introducir datos en base a documentos de txt, enlaces web, imágenes, etc.
  \item Permitir configurar distintos tipos de usuarios.
  \end{enumerate}

\item Analizar la forma más eficiente de desarrollar la aplicación web, teniendo en cuenta la importancia que tendrá la parte visual e interectciva de la misma; así como estudiar de qué forma se ha de guardar y relacionar toda la información que se vaya añadiendo.

\item Documentar las distintas fases del ciclo de vida del software, que se explican en la sección~\ref{sec:metodologia-y-plan}

\end{itemize}


\section{Metodología y plan de trabajo}
\label{sec:metodologia-y-plan}

Estas son las fases de desarrollo que se van a seguir para la
consecución de los objetivos del proyecto descritos en la sección~\ref{sec:objetivos-y-campo}:

\begin{enumerate}
  
\item Requisitos (x mes)
  \begin{itemize}
  \item Identificar exactamente qué se desea mostrar y cómo respecto al pensamiento inicial.
  \item Con esa idea inicial se puede dibujar una idea para diferenciar la separación entre las distintas utilidades que ofrecerá la plataforma.
  \item Identificar las funcionalidades a las que podrá acceder cada usuario.
  \item Determinar cuál es la mejor forma para desarrollar la herramienta, si un software de escritorio que envía y recibe información, una aplicación web multiusuario, etc.
  \end{itemize}

\item Análisis (x meses)
  \begin{itemize}
  \item Se analizan los plazos y tiempos respecto a la planificación del análisis.
  \item Si no se ajustan al tiempo disponible se deberá modificar o adaptar el diseño a los tiempos.
  \item Identificar qué patrón de arquitectura se adec\'ua mejor (MVC o MVVM).
  \item Qué tipo de base de datos es mejor según los datos que van a ser tratados.
  \item Definir los límites y tipos de los distintos usuarios.
  \item Definir las funciones de cada parte de la plataforma y cómo se relacionan.
  \item Definir cómo se van a identificar los distintos sondeos o encuestas a seguir para que sea configurable por cada usuario y empleado por otros para realizar la introducción de información desde distintas fuentes.
  \end{itemize}


\item Diseño (x meses)
  \begin{itemize}
  \item Diseño de la arquitectura general de la plataforma
  \item Diseño particular de las distintas utilidades de la web
  \item Diseño de la base de datos
  \end{itemize}
  
\item Codificaci\'on e integraci\'on (x mes)
  \begin{itemize}
  \item Se crea la base de datos que se había diseñado y se integra con la aplicación web
  \item Se programa toda la aplicación web en base al análisis previo
  \end{itemize}

\item Pruebas (x mes)
  \begin{itemize}
  \item Se probará que todos los módudos funcionan correctamente y que todas las funcionalidades estén presentes
  \end{itemize}

\item Implementaci\'on  (x meses)
  \begin{itemize}
  \item En caso de que se publique.
  \end{itemize}

\end{enumerate}


\section{Medios}
\label{sec:medios}

\textit{Aquí se describen de los medios necesarios para realizar el TFG. Por
  ejemplo:}

Las herramientas que van a ser necesarias para desarrollar este proyecto
son las siguientes:

\begin{itemize}
\item PC compatible
\item Entorno de desarrollo compatible con los últimos framworks de .NET  \ldots
\item Sistema operativo GNU/Linux~\cite{gnulinux}
\item Entorno de desarrollo Emacs/Vim~\cite{emacs}
\item Procesador de textos \LaTeX~\cite{lamport94}
\item Control de versiones CVS~\cite{cvs}
\item Compilador C/C++ gcc~\cite{gcc}
\item Gestor de compilaciones make~\cite{make}
\item Robot con movilidad.
\item  \ldots
\end{itemize}



Otros recursos necesarios para la elaboración del proyecto son:

\begin{itemize}
\item Herramientas  \ldots
\item Sistema de desarrollo  \ldots
\item  \ldots
\end{itemize}




% 
% END Normal sections. Edit/modify all within this section
%%%%%%%%%%%%%%%%%%%%%%%%%%%%%%%%%%%%%%%%%%%%%%%%%%%%%%%%%%%%%%%%%%%%%%%%%%% 
%%%%%%%%%%%%%%%%%%%%%%%%%%%%%%%%%%%%%%%%%%%%%%%%%%%%%%%%%%%%%%%%%%%%%%%%%%% 
%%%%%%%%%%%%%%%%%%%%%%%%%%%%%%%%%%%%%%%%%%%%%%%%%%%%%%%%%%%%%%%%%%%%%%%%%%% 
%%%%%%%%%%%%%%%%%%%%%%%%%%%%%%%%%%%%%%%%%%%%%%%%%%%%%%%%%%%%%%%%%%%%%%%%%%% 
%%%%%%%%%%%%%%%%%%%%%%%%%%%%%%%%%%%%%%%%%%%%%%%%%%%%%%%%%%%%%%%%%%%%%%%%%%% 
%%%%%%%%%%%%%%%%%%%%%%%%%%%%%%%%%%%%%%%%%%%%%%%%%%%%%%%%%%%%%%%%%%%%%%%%%%% 
%%%%%%%%%%%%%%%%%%%%%%%%%%%%%%%%%%%%%%%%%%%%%%%%%%%%%%%%%%%%%%%%%%%%%%%%%%% 


%%%%%%%%%%%%%%%%%%%%%%%%%%%%%%%%%%%%%%%%%%%%%%%%%%%%%%%%%%%%%%%%%%%%%%%%%%% 
% Bibliography
%%%%%%%%%%%%%%%%%%%%%%%%%%%%%%%%%%%%%%%%%%%%%%%%%%%%%%%%%%%%%%%%%%%%%%%%%%% 
\input{../Book/biblio/bibliography.tex}               % EDIT this file if required



\end{document}

